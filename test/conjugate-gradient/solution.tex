\documentclass{article}
\usepackage{graphicx}
\usepackage{subfigure}
\usepackage{fancyhdr}
\usepackage{extramarks}
\usepackage{amsmath}
\usepackage{amsthm}
\usepackage{amsfonts}
\usepackage{tikz}
\usepackage[plain]{algorithm}
\usepackage{algpseudocode}
\usepackage{fancyvrb}
\usetikzlibrary{automata,positioning}

%
% Basic Document Settings
%

\topmargin=-0.45in
\evensidemargin=0in
\oddsidemargin=0in
\textwidth=6.5in
\textheight=9.0in
\headsep=0.25in

\linespread{1.1}

\pagestyle{fancy}
\lhead{\hmwkAuthorName}
\chead{\hmwkClass\ : \hmwkTitle}
\rhead{\firstxmark}
\lfoot{\lastxmark}
\cfoot{\thepage}

\renewcommand\headrulewidth{0.4pt}
\renewcommand\footrulewidth{0.4pt}

\setlength\parindent{0pt}

%
% Create Problem Sections
%

\newcommand{\enterProblemHeader}[1]{
    \nobreak\extramarks{}{Problem \arabic{#1} continued on next page\ldots}\nobreak{}
    \nobreak\extramarks{Problem \arabic{#1} (continued)}{Problem \arabic{#1} continued on next page\ldots}\nobreak{}
}

\newcommand{\exitProblemHeader}[1]{
    \nobreak\extramarks{Problem \arabic{#1} (continued)}{Problem \arabic{#1} continued on next page\ldots}\nobreak{}
    \stepcounter{#1}
    \nobreak\extramarks{Problem \arabic{#1}}{}\nobreak{}
}

\setcounter{secnumdepth}{0}
\newcounter{partCounter}
\newcounter{homeworkProblemCounter}
\setcounter{homeworkProblemCounter}{1}
\nobreak\extramarks{Problem \arabic{homeworkProblemCounter}}{}\nobreak{}

%
% Homework Problem Environment
%
% This environment takes an optional argument. When given, it will adjust the
% problem counter. This is useful for when the problems given for your
% assignment aren't sequential. See the last 3 problems of this template for an
% example.
%
\newenvironment{homeworkProblem}[1][-1]{
    \ifnum#1>0
        \setcounter{homeworkProblemCounter}{#1}
    \fi
    \section{Problem \arabic{homeworkProblemCounter}}
    \setcounter{partCounter}{1}
    \enterProblemHeader{homeworkProblemCounter}
}{
    \exitProblemHeader{homeworkProblemCounter}
}

%
% Homework Details
%   - Title
%   - Due date
%   - Class
%   - Section/Time
%   - Instructor
%   - Author
%

\newcommand{\hmwkTitle}{Homework\ \#2}
\newcommand{\hmwkDueDate}{\today}
\newcommand{\hmwkClass}{MATH6644 Iterative Methods}
\newcommand{\hmwkClassTime}{}
\newcommand{\hmwkClassInstructor}{}
\newcommand{\hmwkAuthorName}{Komahan Boopathy}

%
% Title Page
%

\title{
    \vspace{2in}
    \textmd{\textbf{\hmwkClass:\ \hmwkTitle}}\\
    \normalsize\vspace{0.1in}\small{\hmwkDueDate}\\
    %\vspace{0.1in}\large{\textit{\hmwkClassInstructor\ \hmwkClassTime}}
    \vspace{3in}
}

\author{\textbf{\Large\hmwkAuthorName}}
\date{}

\renewcommand{\part}[1]{\textbf{\large Part \Alph{partCounter}}\stepcounter{partCounter}\\}

%
% Various Helper Commands
%

% Useful for algorithms
\newcommand{\alg}[1]{\textsc{\bfseries \footnotesize #1}}

% For derivatives
\newcommand{\deriv}[1]{\frac{\mathrm{d}}{\mathrm{d}x} (#1)}

% For partial derivatives
\newcommand{\pderiv}[2]{\frac{\partial #1}{\partial #2}}

% Integral dx
\newcommand{\dx}{\mathrm{d}x}

% Alias for the Solution section header
\newcommand{\solution}{\textbf{\large Solution}}

% Probability commands: Expectation, Variance, Covariance, Bias
\newcommand{\E}{\mathrm{E}}
\newcommand{\Var}{\mathrm{Var}}
\newcommand{\Cov}{\mathrm{Cov}}
\newcommand{\Bias}{\mathrm{Bias}}

\begin{document}
\maketitle
\thispagestyle{empty}

\pagebreak

\begin{homeworkProblem}
Consider a matrix
$$A = 
\begin{bmatrix}
  6 & -1 & 0 \\ -1 & 5 & 0 \\ 0 & 0 & 2 \\
\end{bmatrix}$$

\paragraph{a.} \emph{Show that A is positive definite.}
\medskip

The eigenvalues of the matrix $A$ is found using $|A-\lambda
I|=0$. The characterisitic polynomial is $(\lambda^2 -11\lambda +
29)(\lambda-2)=0$. The eigenvalues are $\lambda=2,
\frac{11-\sqrt{5}}{2}, \frac{11+\sqrt{5}}{2}$, all of which are real
(due to symmetry of $A$) and greater than zero (due to positive
definiteness of $A$).

\paragraph{b.} \emph{Calculate the A-norm of the vector.}
$$ \mathbf{b} = 
\begin{bmatrix}
  1 \\ 2 \\ -1
\end{bmatrix} $$
\medskip
The A-norm of vector $||b||_{A} = \sqrt{(b, Ab)}$. We find
$$ \mathbf{c} = \mathbf{Ab} = 
\begin{bmatrix}
  6 & -1 & 0 \\
 -1 &  5 & 0 \\
  0 &  0 & 2 \\
\end{bmatrix} 
\begin{bmatrix}
  1 \\ 2 \\ -1
\end{bmatrix} =
\begin{bmatrix}
  4 \\ 9 \\ -2
\end{bmatrix}
$$
The A-norm of vector $b$ is therefore $\sqrt{(b,c)}=4.899$

\paragraph{c.} \emph{Find a vector that is a orthogonal to $\mathbf{b}$.}

This means that we need to find a vector $d$ that such that $(b, Ab) =
(b, c) = 0$.  We are essentially finding a conjugate vector to $b$. We
can use Gram-Schmidt orthogonalization to get that vector $d$ that is
orthogonal to $c$ as follows:

\end{homeworkProblem}

\clearpage
\begin{homeworkProblem}
 Assume that $A$ is a $n\times n$ symmetric positive definite
  matrix. If $\kappa(A) = {\cal{O}}(n)$, give a rough estimate of the
  number of steepest descent iterations required to reduce the error
  in $A-norm$ to ${\cal{O}}(1/n)$.
  \medskip
\end{homeworkProblem}

\clearpage
\begin{homeworkProblem}
\emph{Assume that $A$ is a $n\times n$ symmetric positive definite
  matrix with eigenvalues in the intervals $(1, 1.1) \cup (10,10.2)$.
  Assume that the cost of a matrix vector multiplication is about $4n$ floating point
  multiplications. Estimate the number of floating point opearations reduce the 
  $A-$norm of the error  by a factor of $10^{-3}$ using CG iterations. }
  \medskip
\end{homeworkProblem}

\clearpage
\begin{homeworkProblem}

\emph{Discretize the following differential equation}
$$ -u^{\prime\prime} = f(x) = 2x -\frac{1}{2}$$
$$ u(x=0) = 1, u^\prime(x=1) = 0, x \in [0,1]$$ \emph{by the central
  difference scheme with $n$ interior mesh points. Solve the
  resulting linear system by Conjugate Gradient (CG), Preconditioned
  CG with Sine transform $([S]_{j,k} = \sqrt{\frac{2}{n+1}}\sin(\frac{\pi j k
  }{n+1}) \forall 1 \le j,k \le n$) to construct the
  preconditioner. Run your code with $n = 64, 128, 256, 512, 1024,
  \ldots$.  Compare the iteration numbers used for each case in a
  table. Discuss your own results.}

  \paragraph{a. Discretization of ODE and Linear System}
  \medskip

  The discretized ODE is $ - u_{i+1} +2 u_i - u_{i-1} =
  h^2f(x_i).$

We impose a Neumann boundary condition that the unknown itself is not
given at the right boundary so we have to solve for it. This means
that our unknowns are not just at the interior points but also at any
point where a Neumann condition is specified. For the sake of
generality, also the Dirichlet condition is achieved
indirectly by including it in the linear system. Therefore, the size
of the linear system is $m=n+2$, where $m$ is the total number of
points and $n$ is the number of interior points.  The resulting linear
system is
 $$ \begin{bmatrix}
 1 &  0 &  0  &  0 & 0 & 0 \\
 -1 & 2 & -1  &  0 & 0 & 0\\
 0 &-1 & 2 & -1  &  0 & 0 \\
0& 0&  \ddots & \ddots & \ddots & 0 \\
0 &0 & 0 & -1 &  2  & -1  \\ 
0 &0 &  0 &  0 &  -1 & 1  \\  
 \end{bmatrix}
\begin{bmatrix}
   u(x_1) \\
    u(x_2) \\
    u(x_3) \\
    \vdots\\
    u(x_{m-1}) \\
    u(x_m) \\
  \end{bmatrix}
  =
  \begin{bmatrix}
    1 \\
    h^2f(x_2) \\
    h^2f(x_3) \\
    \vdots\\
    h^2f(x_{m-1}) \\
    0 \\
  \end{bmatrix}$$
This is a banded tri-diagonal system.



\end{homeworkProblem}


\end{document}


  \begin{figure}[H]
    \centering
    \includegraphics[width=0.65\textwidth]{profile.pdf}    
    \caption{Plot of solution profile in the domain for different
      solution methods.}
    \label{fig:profile}
  \end{figure}
